%\listfiles % für Fehlersuche
% Dokumentenklasse und Optionen
\documentclass[
  final,      % endgültige Fassung
% draft,      % Vorschau
  idxtotoc,   % Index soll im Inhaltsverzeichnis stehen
% liststotoc, % Abbildungs- und Tabellenverzeichnis sollen im Inhaltsverzeichnis stehen
% bibtotoc,   % Literaturverzeichnis soll im Inhaltsverzeichnis stehen
  tocleft,    % Inhaltsverzeichnis nicht einrücken
% chapterprefix, % »Kapitel« vor Kapitelnummer ausgeben (nur scrreprt und scrbook)
% appendixprefix, % »Anhang« vor Anhangsnummer ausgeben
  a4paper,    % Papiergröße A4
  abstracton, % Zusammenfassung aktivieren
  titlepage,  % Eine Titelseite erstellen
% twoside,    % zweiseitiges Layout
  oneside,    % einseitiges Layout
  DIVcalc
]{scrartcl} 

%% ==================================
%% Pakete einbinden
%% ==================================

\usepackage{makeidx} % Paket zur Indexerzeugung
\makeindex % Indexerzeugung aktivieren
%\usepackage[style=altlist,toc=true,hyper=true,number=none]{glossary} % Paket für Glossar
%\makeglossary % Glossarerzeugung aktivieren
%Glossareinträge an der jeweiligen Textstelle mit:
% \glossary{
%   name={Computer},
%   description={Eine Maschine zur Lösung von Problemen,
%     die es ohne diese Maschine nicht gäbe.}
% }
%Erstellung des Glossars mit makeindex:
%makeindex -s dateiname.ist -t dateiname.glg -o dateiname.gls dateiname.glo

\usepackage[headsepline,footsepline]{scrpage2} % Kopf- und Fußzeile
\usepackage[utf8]{inputenc} % Encoding der TeX-Dateien:
% latin1 = Windows,
% ansi = Linux alt,
% utf8 = Linux aktuell
\usepackage[T1]{fontenc} % Aktiviert EC-Schriftarten
\usepackage{ae} % Schöne Schriften für PDF-Dateien
\usepackage{textcomp} % Text-Companion-Symbols (z. B. \texteuro)
\usepackage[ngerman]{babel} % Deutsche Einstellungen
%Paket times ist veraltet, siehe
% http://www.ctan.org/tex-archive/info/german/l2tabu/l2tabu.pdf
% \usepackage{times} % Times-Schriftart auswählen
%Auswahl der Schriftarten:
%
%  -- Palatino --
% Paket mathpazo akzeptiert diese Optionen:
% leer    lieber nicht benutzen
% [sc]    Palatino(pplx) mit echten Kapitälchen
% [osf]   Palatino(pplj) mit echten Kapitälchen und
%         Oldstyle-Zahlen im Text
%\usepackage[sc]{mathpazo}
%\linespread{1.04} % Zeilenabstand passend zur Schriftart erhöen
%\usepackage[scaled=.95]{helvet}
%\usepackage{courier}%
%  -- Times --
%\usepackage{mathptmx} % Times
%\usepackage[scaled=.90]{helvet}
%\usepackage{courier}%
%  -- Zapf Chancery --
%\usepackage{chancery}
%\usepackage[scaled=.90]{helvet}
%\usepackage{courier}%
%  -- Bookman --
%\usepackage{bookman}
%\usepackage[scaled=.90]{helvet}
%\usepackage{courier}%
%  -- New Century Schoolbook --
%\usepackage{newcent}
%\usepackage[scaled=.90]{helvet}
%\usepackage{courier}%
%  -- Charter --
%\usepackage{charter}
%\usepackage[scaled=.90]{helvet}
%\usepackage{courier}%
%  -- Latin Modern --
\usepackage{lmodern}

\typearea[current]{last}
\usepackage{marvosym} % Symbole und das Euro-Zeichen \EUR
\usepackage{eurosym} % Euro-Zeichen nach offz. Vorgaben, Geld durch \EUR{100,00}
\usepackage[final]{microtype} % Mikrotypographische Anpassungen
\usepackage{fixltx2e} % korrigiert LaTeX-Fehler
%\usepackage{mparhack} % wenn verfügbar, korrigiert Fehler bei marginpar-Befehl
\usepackage{color,graphics,graphicx} % Unterstützung für Grafiken und Farben
%\usepackage{picins} % Text umfließt Grafiken
%\usepackage{eso-pic} % nur falls Entwurf-Seitenhintergrund
% \usepackage{url} % nur falls kein hyperref-Paket genutzt wird
% \usepackage{longtable} % für Tabellen über mehrere Seiten
\usepackage{fancyvrb} % für verbatiminput
\usepackage[german]{fancyref}
% \usepackage{listings} % Quelltexte formatiert ausgeben
%\usepackage{booktabs} % Tabellenformatierungsbefehle top-/mid-/bottomrule
\usepackage{caption}
%\usepackage[ngerman]{varioref} % variable Referenzen
%\usepackage[square]{natbib} % Literaturverzeichnis
% PDF-Optionen, PDF-Hyperlinks
\usepackage[draft=false, %
  bookmarks=true, bookmarksnumbered=true, bookmarksopen=true, %
  colorlinks, linkcolor=blue, urlcolor=blue, %
  pdfauthor={Erik Wegner}, pdftitle={Versionsverwaltung mit Subversion}, pdfpagemode=UseOutlines, pdfstartview=Fit, pdfview=FitBH, a4paper]{hyperref}
\usepackage[all]{hypcap} %Klicken auf Verweise rückt Ziel in den Mittelpunkt
%\usepackage{pdfcolmk} %bei Problemen mit Farben beim Seitenwechsel

%veraltete Pakete, die nicht mehr benötigt werden
%t1enc = ersetzt durch fontenc
%times = ersetzt durch mathptmx
%a4    = ersetzt durch Klassenoption a4paper

%% ==================================
%% Standardbefehle für \newcommand
%% ==================================

%%Dieses Kommando trennt die einzelnen Dokumentenabschnitte
%%wahlweise eine neue Seite
\newcommand{\ABSCHNITTSTRENNER}{\cleardoublepage }
%%Dieses Kommando setzt einen Backslash,
%%an dem ohne ein Trennzeichen getrennt werden kann
\newcommand{\trennBS}{\texttt{\char92}" "}

%Einstellungen für Abbildungsbeschriftungen
\captionsetup{margin=10pt,font=small,labelfont=bf,position=bottom}

%%Dieses Kommando setzt um den übergebenen Parameter
%%deutsche doppelte Anführungszeichen
\newcommand{\gdq}[1]{\glqq{}#1\grqq{}}

%%Dieses Kommando setzt um den übergebenen Parameter
%%französische Anführungszeichen (innen, außen, einfach, doppelt)
\newcommand{\fqi}[1]{\frq#1\flq}                 % Anführungsz. einfach
\newcommand{\fqqi}[1]{\frqq#1\flqq}              % Anführungsz. doppelt
\newcommand{\fqa}[1]{\flq#1\frq}                 % Anführungsz. einfach
\newcommand{\fqqa}[1]{\flqq#1\frqq}              % Anführungsz. doppelt

%enquote-Befehl für alle Anführungszeichen verwenden,
%dann kann schnell eine andere Formatierung genutzt werden
%\newcommand{\enquote}[1]{\fqqi{#1}}              % diese Anf.zeichen nutzen
\usepackage[babel,german=guillemets]{csquotes} % falls Paket nicht vorhanden, vorherige Zeile aktivieren

%%Ein Platzhalter für Daten, die später ergänzt werden
%\usepackage[margin,ngerman]{fixme}
\newcommand{\weissnicht}{XX?\-XX?\-XX?\-XX\fixme{Hier fehlt was!}}

%%Der Parameter erscheint im Text und im Index
\newcommand{\textindex}[1]{#1\index{#1}}

%% ==================================
%% Standardfarben definieren
%% ==================================
\definecolor{gruen}{rgb}{0.0,1.0,0.0}
\definecolor{gelb}{rgb}{1.0,1.0,0.0}
\definecolor{weiss}{rgb}{1.0,1.0,1.0}
\definecolor{schwarz}{rgb}{0.0,0.0,0.0}
\definecolor{rot}{rgb}{1.0,0.0,0.0}


%% ==================================
%% Kopf- und Fußzeile definieren
%% ==================================
\manualmark % Manuelle Einstellungen anschalten
\pagestyle{scrheadings}
\ihead{%\includegraphics[height=2ex,draft=false]{zusatz/logo}
}
\chead{}
\ohead{\headmark}

\ifoot{\tiny Name}
\cfoot{}
\ofoot{\pagemark}

% Eine Seite ohne Kopf- und Fußzeile
\defpagestyle{plainwfl}{(0pt,0pt){} {} {}(0pt,0pt)}{(0pt,0pt){} {} {}(0pt,0pt)}

%%Unterste Zeile
\flushbottom % gleiche Höhe, elastische vertikale Abstände
%\raggedbottom % wechselnde Höhe, konstante vertikale Abstände

\begin{document}

%% ==================================
%% Befehle für die Titelseite
%% ==================================
\title{
%\includegraphics[scale=1.0]{zusatz/logo_a} \\
Projektverwaltung mit Subversion
%\includegraphics[draft=false,scale=0.333]{zusatz/logo_a}
}
\author{}
%Datum wird automatisch eingefügt
\pdfbookmark[1]{Titelseite}{titel}
\maketitle
\thispagestyle{plainwfl}

\mbox{}

\vfill


%% ==================================
%% Die Zusammenfassung
%% ==================================
%\begin{abstract}\pdfbookmark[1]{Zusammenfassung}{}

%\end{abstract}

\markleft{}
\markright{Inhaltsverzeichnis}

\ABSCHNITTSTRENNER
\pdfbookmark[1]{\contentsname}{toc}
\tableofcontents
%\listoffigures
%\listoftables
%\listoffixmes

%%%%%%%%%%%%%%%%%%%%%%%%%%%%%%%%%%%%%%%%%%%%%%%%%%
\ABSCHNITTSTRENNER
\markright{Allgemeines}
\section{Versionsverwaltung mit Subversion}

\subsection{Arbeitskopie erstellen}
\begin{Verbatim}
mkdir mac-only 
cd mac-only 
svn checkout https://BENUTZERNAME@banbury.svn.sourceforge.net/svnroot/banbury
\end{Verbatim}


\subsection{Änderungen in das Archiv übertragen}
\begin{Verbatim}
svn update
svn commit --message "Mitteilung über Änderungen"
\end{Verbatim}


\subsection{Verzeichnisse}

\subsubsection{trunk}\label{sec:trunk}
Hauptentwicklungszweig, der Inhalt dieses Verzeichnises wird auf den
Webserver überspielt

\subsubsection{branches}
Ablegen von Nebenentwicklungszweigen für größere, länger dauernde
Änderugen

\begin{Verbatim}
svn copy trunk branches/geek9
\end{Verbatim}

Der Befehl kopiert alle Dateien (und Verzeichnisse) aus dem aktiven
Zweig in den Entwicklerzweig mit der Bezeichnung geek9. In diesem
Nebenzweig können nun umfangreiche Änderungen\footnote{Umfangreich kann
  heißen: viele Dateien betreffend, viele Programmzeilen betreffend,
  eine lange Zeitdauer betreffend, Änderung eines Modells mit
  weitreichenden Folgen, etc.}  vorgenommen werden. Änderungen des
Hauptzweigs lassen sich bei Bedarf übernehmen. Nach Abschluss der
Arbeiten kann der Nebenzweig wieder mit dem Hauptzweig vereint werden.

\subsubsection{tags}
Ablage \enquote{alter} Versionen mit:

\begin{Verbatim}
svn copy trunk tags/v1.1
\end{Verbatim}

Letztlich passiert hier folgendes: alle Dateien aus dem aktiven Zweig
(\fref{sec:trunk}) werden in das Unterverzeichnis v1.1 kopiert. An den
Verzeichnissen und Dateien im Verzeichnis tags dürfen keine Änderungen
vorgenommen werden, da sonst die historische Version abweicht. Erlaubt
ist nur das Anlegen zusätzlicher Tags (Verzeichnisse) mit dem oben
genannten Befehl.


%%%%%%%%%%%%%%%%%%%%%%%%%%%%%%%%%%%%%%%%%%%%%%%%%%
\ABSCHNITTSTRENNER
\appendix
\markright{Anhang}

%%%%%%%%%%%%%%%%%%%%%%%%%%%%%%%%%%%%%%%%%%%%%%%%%%
%\ABSCHNITTSTRENNER
%\markright{Glossar}
%\def\glossaryname{Glossar}
%\printglossary

%%%%%%%%%%%%%%%%%%%%%%%%%%%%%%%%%%%%%%%%%%%%%%%%%%
\ABSCHNITTSTRENNER
\markright{Index}
\renewcommand*{\indexpagestyle}{scrheadings}
\printindex

%%%%%%%%%%%%%%%%%%%%%%%%%%%%%%%%%%%%%%%%%%%%%%%%%%
%\ABSCHNITTSTRENNER
%\markright{Literaturverzeichnis}
%\bibliographystyle{dinat}%[Autor, Jahr]
%\bibliographystyle{gerplain} %einfache Zahl
%\bibliography{dateiname}

\end{document}

% htlatex subversion.tex "xhtml,uni-html4" " -cunihtf -utf8"

